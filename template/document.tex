% Beispiel Dokument
\documentclass[12pt,digital]{hska}

\addbibresource{document.bib}

% Erstellung eines Index und Abkürzungsverzeichnisses/Glossars aktivieren ------------------------------
% \makeindex{}
% \makenomenclature{}

% Das eigentliche Dokument -----------------------------------------------------------------------------
% ------------------------------------------------------------------------------------------------------
\begin{document}

% Titelblatt
% ------------------------------------------------------------------------------------------------------

% vorgegebene Seite verwenden, zB bei Bachelor- oder Masterarbeit:
% ------------------------------------------------------------------------------------------------------
% \thispagestyle{plain}
% \begin{titlepage}
% \includepdf[pages={1}]{sections/coverpage.pdf}
% \end{titlepage}

% einfache Seite selbst schreiben:
% ------------------------------------------------------------------------------------------------------
\thispagestyle{plain}
\begin{titlepage}
\begin{center}
% Logo der HSKA
\includegraphics[width=0.7\textwidth]{images/hka_logo.png}\\[12ex]

\LARGE{\textbf{Seminararbeit}}\\[8ex]

\textbf{Titel der Arbeit\\
ggf. zweizeilig}\\[12ex]

\normalsize{}
\begin{tabular}{lll}
Student:  & \quad Max Muster\\[3ex]
Betreuer:  & \quad Prof. Dr. Patrick Baier &\\[3ex]
Semester: & \quad Sommersemester 2023\\[3ex]
\end{tabular}

\end{center}
\end{titlepage}

% Seitennummerierung -----------------------------------------------------------------------------------
%   Vor dem Hauptteil werden die Seiten in großen römischen Ziffern nummeriert.
% ------------------------------------------------------------------------------------------------------
\pagenumbering{Roman}
\phantomsection{} % Sorgt für korrekte Aufnahme des Inhaltsverzeichnisses in das Inhaltsverzeichnis
\addcontentsline{toc}{section}{Inhaltsverzeichnis}
\tableofcontents

% arabische Seitenzahlen im Hauptteil ------------------------------------------------------------------
\clearpage{}
\pagenumbering{arabic}
\setcounter{page}{2} %%% Dieser Pagecounter muss entsprechend der verbrauchten Seiten im Inhaltsverzeichnis angepasst werden. Endet das IHV bei Seite III, so muss hier 4 eingetragen werden



% Inhalt -----------------------------------------------------------------------------------------------
% ------------------------------------------------------------------------------------------------------

% Kapitel sinnvollerweise in eigene Dateien schreiben und einbinden
\section{Introduction}
Lorem ipsum dolor, sit amet consectetur adipisicing elit.
Quis deleniti, velit distinctio qui inventore id aliquid illum,
sunt perspiciatis unde dicta iusto, officiis eius eos fuga possimus sit omnis ratione?
\cite{logo_hska}asdf
% Neue Seite nach Kapitel
\clearpage{}



% Literaturverzeichnis ---------------------------------------------------------------------------------
%   Das Literaturverzeichnis wird ggf. aus der BibTeX-Datenbank "document.bib" erstellt.
% ------------------------------------------------------------------------------------------------------
% Quelle ins Quellenverzeichnis aufnehmen, die im Dokument nicht genutzt wird.
\nocite{logo_hska}
\renewcommand\refname{Quellen}
\printbibliography



% Anhang -----------------------------------------------------------------------------------------------
%\begin{appendix}
 %   \clearpage{}
  %  \pagenumbering{roman}
%	\section{Anhang}
%\end{appendix}

\end{document}